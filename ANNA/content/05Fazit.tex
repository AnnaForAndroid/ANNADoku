\chapter{Fazit}
Bei der Entwicklung der Applikation \ac{ANNA} spielten sowohl Erfahrungen als auch neue Erkenntnisse eine zentrale Rolle.\\
Durch den Vergleich mit bestehenden Produkten kristallisierte sich ein guter Fokus für die Entwicklung der Anwendung heraus. Zusätzlich wurden durch die immer wieder durchgeführten Tests der Applikation in realistischen Nutzungsumgebungen Schwachstellen erkannt und diese umgehend behoben. So wurde beispielsweise als Ergebnis eines solchen Tests die Benutzeroberfläche von Grund auf neu gestaltet. So wurde sichergestellt, dass die einzelnen Elemente auch auf einer mittleren Distanz, wie sie während der Fahrt herrscht wenn das Smartphone sich in einer entsprechenden Halterung befindet, gut zu erkennen sind.

Auch konnten während der Entwicklung Erfahrungen zum Umgang mit Spracherkennugsframworks und deren Probleme hinsichtlich der Verwendung in einer Anwendung wie \ac{ANNA} gesammelt werden.\\
Beispielsweise stellte sich so die Problematik der Verwendung der Spracherkennung während der Nutzung eines Musik-Players heraus.

In Anbetracht des begrenzten Entwicklungszeitaums, konnten einige Funktionen nur in begrenzten Umfang umgesetzt werden, woraus sich die in Kapitel \ref{ausblick} beschriebenen weiteren Entwicklungsspielräume ergeben.\\
Dennoch ist mit dem Ergebnis dieser Arbeit bereits eine Lösung entstanden, mit deren Hilfe sich die Ablenkung im Straßenverkehr durch Smartphones reduzieren lässt.

Generell lässt sich sagen, dass durch die Entwicklung dieses Projekts viele neue Erkenntnisse zu mobilen Plattformen, Spracherkennungsfunktionen, Mensch-Computer Interaktionen und der Gestaltung von Benutzeroberflächen sammeln ließen. Außerdem wurde ein nicht unwesentlicher Beitrag zur Reduzierung von Verkehrsunfällen geleistet, welcher im besten Fall sogar Menschenleben retten kann. 