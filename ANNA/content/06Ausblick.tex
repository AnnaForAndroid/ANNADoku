\chapter{Ausblick}
\label{ausblick}
Die Entwicklung des Projekts \ac{ANNA} hat bereits einen guten Stand erreicht, jedoch gibt es noch Verbesserungspotential für die zukünftige Entwicklungsarbeit.

Aus technischer Sicht kann die Applikation zukünftig verbessert werden, indem Optimierungen an der Performance vorgenommen werden.\\
Einige Funktionen die durch die Applikation bereitgestellt sind äußerst Zeitaufwändig. Eine mögliche Verbesserung wäre es an dieser Stelle die entsprechenden zeitkritischen Programmteile zu parallelisieren, um so eine bessere Geschwindigkeit bei der Erfüllung der gebotenen Funktionen zu erreichen und dem Benutzer ein flüssigeres Nutzererlebnis zu liefern.\\
Als konkretes Beispiel für einen solch zeitkritischen Programmteil ist hier die Einrichtung der Applikation nach dem erstmaligen Start abzuführen.

Ein weiteres Verbesserungspotential verbirgt sich hinter den Spracherkennungsfunktionalitäten.\\
Da das bei der Entwicklung verwendtete Framework, \ac{CMU} in erster Linie für die englische Sprache entwickelt wird, ist die Anzahl der im phonetischen Wörterbuch enthaltenen Wörter der deutschen Sprache eher gering. Hieraus ergibt sich der Bedarf der Erweiterung des verwendeten Wörterbuchs, um so die Spracherkennung zu verbessern.\\
Zusätzlich wäre es möglich den Nutzer die verwendeten Grammatiken personalisieren zu lassen, so dass sich für die Verwendung der Spracherkennungsfunktionen ein natürlicheres Konversationsgefühl ergibt. Zusätzlich würde diese Konfiguration die Möglichkeiten der personalisierung der Applikation weiter erhöhen.\\
Denkbar wäre es hier zum Beispiel, dass dem Nutzer der Sprachbefehl "navigiere mich nach...", welcher zum starten der Turn-by-Turn Navigation verwendet wird, nicht gefällt. Durch die neue Konfigutarionsmöglichkeit könnte er den Befehl dann zum Beispiel zu "bring mich nach..." umwandeln.

Aus Funktionsumfangtechnischer Sicht ergeben sich für zukünftige Entwicklungen die Möglichkeit mehr Dienste in die Applikation zu integrieren. Möglich wäre es hier beispielsweise aus Sicht der Messenger Anwendungen die Liste der unterstützten Messenger zu erweitern und ebenfalls nicht nur die Möglichkeit zur Antwort auf einen Chatt zu bieten sondern auch die Möglichkeit einen neune Chatt zu initialisieren.\\
Auch aus Sicht der Musik-Diensten würden sich weitere Dienste implementieren lassen und so wäre es möglich dem Nutzer noch wesentlich größere Personalisierungsmöglichkeiten zu bieten.

Auch bei dem angebotenen integrierten Navigationsdienst besteht die Möglichkeit noch weitere Dienste in die Applikation zu implementieren und die durch den Navigationsdienst bereitgestellten Funktionen zu erweitern.\\
Beispielsweise ist es denkbar, dass in Zukunft neben der reinen Turn-by-Turn Navigation zusätzliche nützliche Funktionen wie beispielsweise eine Blitzerwarnung, die Anzeige von Geschwindigkeitslimits oder das suchen nach \ac{POI} zur Verfügung stehen.\\
Außerdem wäre es denkbar den Nutzer die Darstellung der Navigationfunktion, über den Einstellungsbereich der Applikation, stärker an seine Vorlieben anzupassen. Zum Beispiel wäre es so für den Nutzer möglich die Ansicht zwischen 2D und 3D zu wechseln.
